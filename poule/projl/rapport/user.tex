\documentclass[french,12pt]{article}
\usepackage{babel}
\usepackage{a4}
\usepackage{amsmath}
\usepackage{graphicx}

\title{POULE}
\author{Guide Utilisateur}

\begin{document}
\sloppy
\maketitle

\strut\thispagestyle{empty}
\vfill
\pagebreak
\setcounter{page}{1}
\tableofcontents
\pagebreak

\section {Installation}

L'installation de  \emph{Poule} est un exercice  facile. Si vous  suivez \`a 
la lettre   les   consignes   delivr\'ees   ici,   vous   n'aurez   pas   de
probl\`emes. Premierement vous devez d\'ecompresser la tarball comme ceci:
\begin{verbatim}
tar zxvf poule-x.x.tar.gz
\end{verbatim}

Ensuite  vous  devez  lancer  le  script  de  configuration  qui  vous
permettra de specifiez la chemin de vos diff\`erentes librairies.

\begin{verbatim}
cd poule-x.x
./configure
\end{verbatim}

En cas de probleme, faites

\begin{verbatim}
./configure --help
\end{verbatim}

Ensuite,  il vous  suffit de  compiler \emph{Poule}.  Vous devez  avoir 
Ocaml version  3.05 au  minimum,  install\'e sur  votre  platforme. Pour  
cela faites

\begin{verbatim}
cd src
make opt
\end{verbatim}

D\'esormais, \emph{Poule} est install\'e. 

\section {Prise en main}

Pour lancer \emph{Poule}, rien de  plus simple. Il vous suffit de vous
trouver  dans le  repertoire source  de  la tarball  et d'\'executer  le
programme.

\begin{verbatim}
./poule
\end{verbatim}

Ensuite, une ligne de commande apparait et il vous suffit de taper vos
commandes en  vous reportant  \`a la grammaire  de \emph{Poule}.  
Voici quelques exemples simples:

\begin{verbatim}
Poule: f = (x : R |-> x^2 + sin(x) : R);
f = x: R |-> ((x^2) + sin(x)): R
Poule: f(5.0) + f(y*2.0);
(sin(5) + (sin((y * 2)) + (((y * 2)^2) + (5^2))))
Poule: a = 2.0;  
a = 2
Poule: f(a+pi);;
20.0200708086
Poule: 
\end{verbatim}

Ici,  nous d\'eclaront  une fonction  $f$ qui  prend un  r\`eel et  qui se
calcul en $(x^2)  + sin(x)$. Nous \'evaluons cette  fonction en $5.0$ et
en $y * 2.0$. Puis nous  affectons une variable d'environnement \`a la valeur
2.0. Enfin nous appliquons $f$ a $2.0+\pi$

\section {Fonctionnalit\'es}

Actuellement les fonctionnalit\'es de \emph{Poule} sont:


\subsection {derivate}

\begin{verbatim}
derivate  (<fnct>, <variable de derivation  ...)  
\end{verbatim}

Cette fonction  d\'erive fcnt  en fonction de  la ou des  variables de
derivation.

\section {Options}

Une option --latex peut \^etre pass\'ee \`a \emph{Poule} comme ceci:

\begin{verbatim}
./poule --latex
\end{verbatim}

Ceci permet une sortie latex  que vous pouvez r\'ecuperer et placer dans
un fichier latex.

Une option --inline qui permet de prendre une formule \`a \'evaluer 
directement en argument.

\begin{verbatim}
./poule --inline ``derivate(y+sin(x*y),x);''
\end{verbatim}

\section {Foire aux questions}
Q: Monsieur, je suis un peu con, Comment ca marche \emph{Poule} ?

R: Lis les section precedentes.

Q: Monsieur, ./configure marche pas ->

akim demaille: Lis config.log

Q: Monsieur, ... ?

R: mail jean-daniel.michaud@epita.fr

\end{document}