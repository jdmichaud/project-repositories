\documentclass[french,12pt]{report}
\usepackage{babel}
\usepackage{a4}
\usepackage{amsmath}
\usepackage{graphicx}

\begin{document}
\sloppy

\begin{titlepage}
\begin{center}

%\setlength{\unitlength}{1in}
%\begin{picture}(0,0)
%   \put(-4.3,-10){\includegraphics{fond4.eps}}
%\end{picture}

.
\vspace{5.3cm}

\Huge \bf {POULE}

\vspace{2cm}

\large
\mdseries
\begin{tabular}{l r}
Nicolas \textsc{Metais} & Jean-Daniel \textsc{Michaud} \\
Jonathan \textsc{Mimouni} & Francois \textsc{Morlot}
\end{tabular}

\vspace{2cm}
Le \today

\end{center}
\end{titlepage}

\strut\thispagestyle{empty}
\vfill
\pagebreak
\setcounter{page}{1}
\tableofcontents
\pagebreak

\section*{Introduction}
\addcontentsline{toc}{chapter}{Introduction}
\markboth{\uppercase{Introduction}}
{\uppercase{Introduction}}

Ce document  est le  cahier des charges  du projet \emph{Poule}.  Il a
pour  principale fonction  de d\'efinir  les limites  de \emph{Poule},
ainsi que de d\'efinir  ses principales fonctionnalit\'es. Nous sommes
rest\'es  d\'elib\'er\'ement flous  dans la  d\'efinition  de celui-ci
afin de ne  pas hypoth\`equer nos chances de  rendre un projet complet
et finalis\'e dans les temps.

Nous  pr\'eciserons aussi  dans  ce document  les  d\'etails de  notre
organisation.  Ainsi la r\'epartition  des t\^aches  et l'organisation
temporelle sont pr\'esent\'ees dans la deuxieme partie.

\chapter{Sp\'ecifications}

Les  math\'ematiques sont  la base  de beaucoup  d'autres  sciences et
applications aussi  bien scientifiques que  commerciales. La physique,
la biologie ou encore la cryptographie, toutes ces sciences se servent
all\`egrement des math\'ematiques  pour formaliser et structurer leurs
th\'eories   fondamentales.  D\`es  l'apparition   de  l'informatique,
domaine qui  comme son  nom l'indique a  pour base  l'information, les
ordinateurs  ont  \'et\'e  mis   \`a  contribution  pour  traiter  des
informations   purement  num\'eriques   en  premier   lieu   pour  les
universit\'es americaines.  Mais  tr\`es vite, les math\'ematiciens ce
sont apercus que les ordinateurs pouvaient bien plus que manipuler des
nombres, aussi grands soient-ils.

Les  calculs formels sont,  en effet,  parfois de  vrais casse-t\^etes
pour les math\'ematiciens m\^eme les plus chevronn\'es. Parfois tr\`es
compliqu\'es, ils sont sources d'erreurs et de pertes de temps. Ainsi,
utiliser   la   puissance   des   ordinateurs   coupl\'ee   \`a   leur
infaillibilit\'e semblait  naturelle.  Mais les choses ne  sont pas si
\'evidentes.   La   puissance  des   ordinateurs   provient  de   leur
simplicit\'e. Amen\'es \`a ne manipuler que des nombres, ils sont tout
\`a  fait  incapables  d'int\'egrer  des notions  de  formalisme.  Une
\'equation  semblant simplissime pour  un \^etre  humain ne  l'est pas
pour un ensemble  de fils et de commutateurs.  Voici un exemple simple
mais convainquant:
$$ f(x) = 2 * x $$

Pour un \^etre humain n'ayant pas oubli\'e ses cours du coll\`ege, que
cela  represente-t-il  ?  Une  fonction.  Une fonction  qui  prend  en
param\`etre  un x  quelconque  et  qui a  pour  valeur le  param\`etre
multipli\'e   par  l'entier   2.    Pour  un   ordinateur,  que   cela
repr\'esente-t-il ? Une simple  cha\^ine de caract\`eres. Une cha\^ine
o\`u  tous  les  symboles  ont  une  m\^eme  signification.   $f$  est
\'equivalent  \`a $)$  qui lui  m\^eme  est \'equivalent  \`a $*$.  On
comprend pourquoi il sera difficile de faire comprendre \`a la machine
que $ f(3) = 6 $.

Le premier  compilateur fut \'ecrit en  avril 1957 par  John Backus et
son \'equipe. Il compilait le  langage \emph{FORTRAN}, mais cela a peu
d'importance. Ce qui  est important, c'est que pour  la premi\`re fois
de  son histoire,  l'ordinateur \'etait  capable de  ``comprendre'' un
langage de  plus haut  niveau que le  simple assembleur,  reflet trait
pour trait, de son  architecture. Le domaine d'application du principe
des  compilateurs  est tr\`es  vaste.  Interpretation  de scripts,  de
fichiers de  configuration, de protocoles  de communication... partout
o\`u l'information est repr\'esent\'ee  sous forme de texte \'ecrit et
lisible  par un  \^etre  humain  et devant  \^etre  manipul\'e par  un
ordinateur. C'est le  cas du calcul formel o\`u la  base du calcul, la
formule,  doit \^etre integr\'ee  et manipul\'ee  par la  machine.  Le
principe d'un  compilateur est assez  simple. A partir  d'une entr\'ee
(du texte) et d'une grammaire,  le compilateur construit un arbre. Cet
arbre est ensuite facilement manipuler par la machine pour produire le
r\'esultat attendu,  que ce soit  pour le traduire en  langage machine
(compilation), pour le parcourir  et le r\'eduire (calcul num\'erique)
ou  le   r\'eduire  et   le  transformer  (calcul   num\'erique).  Ces
op\'erations sont les objectifs du projet \emph{Poule}.

\section{But}

Notre but  est de r\'ealiser un  logiciel de calcul  formel capable de
manipuler  une  formule entr\'ee  par  l'utilisateur et  d'\'effectuer
diff\'erents  calculs de mani\`ere  formelle. De  m\^eme, \emph{Poule}
sera capable  d'\'effectuer des  calculs num\'eriques sur  des nombres
arbitrairement grands, la capacit\'e ne serait alors limit\'ee que par
les  caract\`eristiques   de  la  machine.    L'analyse  de  fonctions
math\'ematiques pourra  \^etre continu\'ee d'une  visualisation par un
module  de  trac\'e  de  courbes.  Afin  de  rendre  l'utilisation  de
\emph{Poule}  plus intuitive,  une interface  permettra la  saisie des
donn\'ees et l'affichage des  formules. Enfin, un langage de scripting
pourrait   venir  completer  les   fonctionnalit\'es  du   projets  et
permettrait l'automatisation de certaines t\^aches.

\section{Module formel}

\emph{Poule}  est avant  tout une  plateforme de  calcul  formel.  Son
environement   permettra  de   manipuler   des  entit\'es   abstraites
d\'efinies par leur appartenance \`a un domaine de d\'efinition.


Constituant  la  base  du   module  calcul  formel,  les  domaines  de
d\'efintion feront l'objet d'un contr\^ole rigoureux qui pr\'ec\'edera
tout calcul.

\subsection{Le contr\^ole de types}

Le contr\^ole  de type constitue la  premi\`ere \'etape d'\'evaluation
d'une expression.


Toutes les entit\'es de \emph{Poule} sont typ\'ees:
\begin{itemize}
\item les valeurs num\'eriques, par d\'efinition
\item les  variables, qui appartiennent  obligatoirement \`a un  et un
  seul domaine de d\'efinition
\item  les op\'erateurs,  qui manipulent  des op\'erandes  typ\'ees et
renvoient des expressions typ\'ees
\item les  fonctions dont  les arguments et  la valeur de  retour sont
typ\'es
\item les types abstraits d\'efinis par l'utilisateur, puisqu'ils sont
form\'es \`a partir de types pr\'eexistants
\end{itemize}

Ce  rigoureux  contr\^ole  du   typage  \`a  pour  but  d'assurer  une
coh\'erence des r\'esultats, de  renforcer la stabilit\'e du projet et
d'apporter un confort \`a l'utilisateur  qui en cas d'erreur de typage
verra  son  op\'eration abandonn\'ee  avant  le  d\'ebut des  calculs,
\'eventuellement co\^uteux en temps.

Une fois le contr\^ole de  type effectu\'e, le moteur de calcul formel
de \emph{Poule} met en place ces diff\'erentes fonctionnalit\'es.

\subsection{Calcul arithm\'etique sur les entit\'es formelles}

La  fonctionnalit\'e de base  du projet  est le  calcul arithm\'etique
avec les operateurs usuels (+, -, *, /) sur des entit\'es formelles.

\paragraph{}
{\bf Exemple: }
\begin{math}
  (x + x + 2)
\end{math}
{\it  est traduit en } 
\begin{math}
  (2 \times x + 2)
\end{math}

\subsection{Factorisation/D\'eveloppement}

La  factorisation  d'une expression  permet  de traduire  l'expression
originale sous  forme d'un  produit d'expressions plus  simples. Cette
fonctionnalit\'e sera  utile pour  les \'etudes de  signe, ou  pour la
r\'esolution d'\'equations.

L'op\'eration   compl\'ementaire    est   le   d\'eveloppement   d'une
expression.

Il est possible de  r\'ealiser une factorisation ou un d\'eveloppement
partiel  en   pr\'ecisant  la   variable  par  rapport   \`a  laquelle
l'op\'eration doit \^etre r\'ealis\'ee.
    
\paragraph{}
{\bf Exemples: }
\begin{enumerate}
\item
  \begin{math}
    fact ((x + 1) * (2 * x + 4))
  \end{math}
  {\it  est traduit en }
  \begin{math}
    (2 \times (x + 1) \times (x + 2))
  \end{math}
\item
  \begin{math}
    expand (x * (5 * y + z) / 7, x)
  \end{math}
  {\it  est traduit en }
  \begin{math}
    (\frac{5 \times x \times y + x \times z}{7})
  \end{math}
\item
  \begin{math}
    fact (expand ((x + 1) * (x + 1))) 
  \end{math}
  {\it  est traduit en }
  \begin{math}
    (x + 1)^{2}
  \end{math}
\end{enumerate}

\subsection{Simplification d'expressions}

La  simplification  d'une  expression  consiste  en  son  \'evaluation
formelle (permettant de simplifier par exemple y - y en 0), puis en sa
factorisation, et  enfin en la simplification  en \'el\'ements neutres
des  expressions  mettant en  relation  une  loi  de composition,  une
op\'erande et son inverse par la  loi de composition (par exemple, 3 *
x / 3 devient x).  Plusieurs algorithmes de simplification existent et
se combinent  tels les simplifications  par termes ou  facteurs. Assez
simples   et   donc   assez   rapides,  ils   donnent   d'assez   bons
r\'esultats.   Mais  il   est   important  de   noter  que   certaines
simplifications  faites  naturellement  \`a  la  main  peuvent  \^etre
dangereuses si elles sont effectu\'ees systematiquement.

Il est aussi int\'eressant  de remarquer que certaines simplifications
donnent des r\'esultats diff\'erents  selon le domaine de d\'efinition
des variables.

\paragraph{}
{\bf Exemples: }
\begin{enumerate}
\item
  \begin{math}
    \sqrt {x^{2}} | x > 0
  \end{math}
      {\it  se simplifie en } x
\item
  \begin{math}
    \sqrt {x^{2}} | x < 0
  \end{math}
  {\it  se simplifie en } -x
\end{enumerate}


\subsection{Substitution de variables}

Il est possible  de substituer une variable par  une valeur, une autre
variable ou une expression.

La substitution d'une  variable par une autre variable  n'est pas sans
rappeler l'alpha-r\'eduction  du lambda-calcul. Cette fonctionnalit\'e
sera se montrer utile  lors de la manipulation d'expressions contenant
des variables libres.

\paragraph{}
{\bf Exemple: }
\begin{math}
  (ln(x) + y) | x = 2 
\end{math}
 and 
\begin{math}
  y = ln(z)
\end{math}
{\it  est traduit en } 
\begin{math}
  (ln(2 \times z))
\end{math}


\subsection{Evaluation d'expressions}

L'\'evaluation d'une expression consiste  en la substitution de toutes
les variables de l'expression  par une valeur respectant l'ensemble de
d\'efinition de la variable, puis en son calcul alg\'ebrique.

\paragraph{}
{\bf Exemple: }
\begin{math}
  (sqrt(x) + y) / z | x = 5
\end{math}
and 
\begin{math}
  y = 1
\end{math}
and 
\begin{math}
  z = 2
\end{math}
{\it  s'\'evalue en } 
\begin{math}
  1.618 033 989
\end{math}

\subsection{R\'esolution de syst\`emes d'\'equations et d'in\'equations}

\emph{Poule} r\'esoud les \'equations  \`a une ou plusieurs inconnues.
Il permet \'egalement la r\'esolution des syst\`emes d'\'equations \`a
solutions r\'eelles et complexes.

Il est  possible de  pr\'eciser un intervale  dans lequel  les racines
(valeurs   des  variables  qui   annulent  les   \'equations)  doivent
appartenir.

Toutes  ces  fonctionnalit\'es  sont  \'egalement  valables  pour  les
in\'equations et syst\`emes d'in\`equations.

\subsection{Calcul de d\'eriv\'ees}

Le moteur de calcul formel  de \emph{ Poule} offre la possibilit\'e de
calculer,   sous   sa  forme   factoris\'ee,   la  d\'eriv\'ee   d'une
fonction. \\Il est \'egalement possible:
\begin {itemize}
\item de calculer la d\'eriv\'ee d'une fonction en un point;
\item de pr\'eciser  la variable par rapport \`a  laquelle doit \^etre
calcul\'ee la d\'eriv\'ee, dans  le cadre d'une fonction \`a plusieurs
variables;
\item d'obtenir la d\'eriv\'ee n-i\`eme d'une fonction.
\item d'effectuer des calculs de Laplaciens, de rotationnels, ...
\end{itemize}


Evidement,   toutes  ces   fonctionnalit\'es  n\'ecessitent   que  les
fonctions utilis\'ees soient d\'erivables.


\subsection{Calcul de primitives et d'int\'egrales}

De  la  m\^eme fa\c{c}on  que  les  d\'eriv\'ees,  \emph{ Poule}  sait
calculer  les  primitives  d'une  fonction.  Il est  possible  de  lui
pr\'eciser la valeur de la constante d'int\'egration.


Li\'e au calcul de primitives, le calcul d'int\'egrales met en jeu des
r\`egles  plus fines  (comme la  reconnaissance d'une  fonction paire,
simplifiant  le r\'esulat).  D'autre part,  \emph{ Poule}  est capable
d'effectuer des  calculs sur des  int\'egrales impropres (int\'egrales
qui font intervenir  des bornes infinies, ou encore  une fonctions non
d\'efinies aux bornes ou en points de l'intervale \'etudi\'e).  Enfin,
l'imbrication  des  int\'egrales  permet  des  calculs  d'int\'egrales
doubles, triples, ...


\section{Module num\'erique}

\subsection{Calcul sur les flottants}

Cette partie regroupe toutes les differentes fonctions appel\'ees lors
de l'\'evaluation  num\'erique d'une formule.  On y trouve,  bien sur,
les  differents  operateurs($+$,$-$,$*$,$/$),  mais aussi  des  autres
fonctions     comme    log,     exponentiel    et     les    fonctions
trigonometriques.    Celles-ci   seront    calcul\'ees    \`a   partir
d'approximations. En effet, il serait interressant de pouvoir disposer
d'une  precision   variable.  On  ne  veut  pas   toujours  un  calcul
extremement  pr\'ecis, mais  il est  utile de  pouvoir  disposer d'une
precision  arbitrairement grande.  Cela nous  permettra  d'inserer des
algorithmes  de calcul de  $\pi$ et  de $e$  pour correspondre  \`a la
pr\'ecision demand\'ee.   Il faudra donc  un module de calcul  sur des
nombres   flottants    arbitrairement   grands,   un   d'approximation
num\'erique  de  formules  et   un  de  calcul  des  constantes  comme
$\pi$. Ces  fonctions devront  respecter une contrainte  de rapidit\'e
car elles  seront utilis\'ees tr\`es souvent. Par  exemple, pour faire
une integrale num\'erique, il faudra calculer des milliers de fois une
fonction complexe contenant elle-m\^eme plusieurs fonctions de calcul.

\subsection{Calcul sur les entiers}

Cette  partie regroupe  toutes  les operations  sp\'ecifiques sur  les
entiers.   En effet,  les operations  se rapproche  de celles  sur les
flottants (sans  aucun chiffre  apres la virgule).  Il y a  donc, dans
cette  partie,  les  differentes  operations  arithmetiques.  On  peut
imaginer differents  tests de primalit\'e (par exemple  un test rapide
probabiliste  et  un  plus  lent  bas\'e sur  le  principe  du  crible
d'erathostene). De plus, on peut rajouter des fonctions de congruence,
de factorisation en facteurs premiers et quelques autres fonctions.


\section{Module graphique}

Le  module graphique  fait partie  des modules  qui ne  presentent \`a
priori que peu  d'interet au niveau de leur  programmation. Mais, tout
comme le module IHM, il a pour r\^ole "d'\'eclairer" l'utilisateur sur
le r\'esultat. En effet,  la tra\'e graphique permet \`a l'utilisateur
de visualiser \'enormement de choses en un seul coup d'oeil; ainsi une
fonction obscure comme par exemple
\begin{center}
\verb!integrate(exp(2*x^3 - 4) + 7) * cos (x)^3!
\end{center}

s'interprete  beaucoup mieux,  pour une  etude rapide,  sous  la forme
d'une courbe.  En effet la representation graphique en 2D d'une simple
fonction donne \`a tout bachelier une mine d'informations: Son allure,
ses racines simples, ...

Bien entendu, nous  avons pour objectif de d\'epasser  ce simple cadre
de  visualisation  en  2D.  Des  rendus 3D  permettant  de  visualiser
certaines courbes  sont prevus. De plus,  une certaine interactivit\'e
avec  le  module  graphique  est  envisageable.  La  possibilit\'e  de
d\'efinir  les bornes  pour le  calcul de  racines directement  sur le
graphe  ou le  calcul d'une  int\'egrale num\'erique  font  partie des
actions rendant le graphe plus dynamique.

\section{Module IHM}

L'Interface  Homme-Machine (\emph{IHM})  de \emph{Poule}  est,  \`a la
base,  constitu\'ee  du  terminal  classique  d'\emph{Unix}.  Celui-ci
permettra la lecture des  formules et des diff\'erents traitements \`a
y appliquer ainsi que  les ``commandes syst\`eme'' aff\`erentes \`a la
gestion de \emph{Poule}. Il  permettra aussi la sortie des r\'esultats
des  calculs   et  des   diff\'erents  messages  \`a   l'attention  de
l'utilisateur.

Produire une  belle interface,  haut en couleur  n'est pas le  but, ni
m\^eme un inter\^et  de ce projet. Malgr\'e tout,  il nous faut \^etre
pragmatique, l'\'ecriture  et la lecture  de formules math\'ematiques,
parfois longues et  complexes peut se transformer en  un vrai calvaire
sur les terminaux usuels  d'\emph{Unix}. C'est pourquoi une \emph{IHM}
graphique,  peut \^etre d'un  grand secours.  Mais comme  nous l'avons
dit, elle  ne doit pas monopoliser  notre temps et  nous emp\^echer de
developper les parties centrales  de notre projet. Ainsi, nous restons
d\'elib\`erement hypoth\'etique  sur l'avenir de  cette \emph{IHM}. Le
but sera  de mod\'eliser un projet  pouvant, si le  temps nous manque,
accepter plus tard un front-end graphique.

De quoi ce  module peut-\^etre constitu\'e ? D'une  part, et cela nous
semble  essentiel, d'une  fen\^etre graphique  capable  d'afficher les
diff\'erents signes math\'ematiques  composant les formules. Chiffres,
lettres, symboles d'int\'egration, etc ... Mais aussi d'un ensemble de
boutons accessibles  \`a la souris permettant la  simplification de la
saisie  de  formules.  Ainsi,  vouloir d\'efinir  l'int\'egral  de  la
fonction  exponentielle   de  borne   inferieure  \'egale  \`a   0  et
sup\'erieure \'egale \`a 1:
$$ \int_{0}^{1} e^x dx $$
se fera difficilement sur un terminal:
$$ \verb!integrate (exp(x), x, 0, 1)! $$
Mais beaucoup plus facilement gr\^ace \`a l'\emph{IHM}. Le proc\'ed\'e,
m\^eme s'il n'est pas rigide, pourra se pr\'esenter ainsi:


$$ \verb!clic sur le bouton d'integration! $$
$$ \verb!clic sur la fonction exponetielle! $$ 
$$ \verb!clic sur la variable x! $$
$$ \verb!entree manuelle des deux bornes! $$
$$ \verb!clic sur le bouton de calcul numerique! $$

On le  voit, l'\emph{IHM}, malgr\'e  son manque d'inter\^et  au niveau
d\'veloppement,  reste un  attribut essentiel  de  \emph{Poule}, c'est
pourquoi son int\'egration future devra \^etre prise en compte.


\chapter{Organisation}
\section{R\'epartition des t\^aches}
\section{Planning}
\section{Prochainement}

La mod\'elisation est la prochaine \'etape de notre travail. Bien nous
organiser  pour construire une  mod\'elisation \`a  la hauteur  de nos
amibtions est l'objectif courant.

\chapter{Sources d'informations}

Principalement  les livres  et  le net.  Notre  souteneur aussi,  Yann
Regis-Gianas. Mais nous envisageons de contacter les auteurs de COQ ou
FOC si le besoin s'en fait sentir.

\section{Internet}

\begin {itemize}
\item {ocaml.inria.fr}
\end {itemize}

\section{Livres}

Les principaux ouvrages sont:
\begin {itemize}
\item {Le calcul formel en pascal, d'Albert Levinas}
\item {Le calcul formel, de ???}
\end {itemize}

\pagebreak

\section*{Conclusion}
\addcontentsline{toc}{chapter}{Conclusion}
\markboth{\uppercase{Conclusion}}
{\uppercase{Conclusion}}

Le grand  d\'efi de notre projet  ne tient pas en  la r\'ealisation de
toutes ces fonctionnalit\'es mais  avant tout en l'\'elaboration d'une
plateforme    de   calcul    formel   {\bf    extensible}    et   {\bf
intelligente}.  Par ``extensible'', nous  entendons qu'il  sera facile
d'int\'egrer   de   nouvelles   fonctionnalit\'es   au   projet.   Par
``intelligente'',  nous  entendons  que  \emph{  Poule}  sera  capable
d'apprendre de l'utilisateur. \\En effet, il sera possible d'enseigner
\`a \emph{ Poule} de nouvelles lois de simplification (par exemple:
\begin{math}
  {x^{a}} \times {x^{b}} = {x^{a+b}}
\end{math}
).  Toute la  puissance de  \emph{  Poule} sera  alors d'utiliser  ces
nouvelles r\`egles assimil\'ees au sein de ses fonctionalit\'es.

\end{document}
